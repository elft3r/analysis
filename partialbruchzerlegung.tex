\section{Partialbruchzerlegung}
Mit dieser Methode wird ein schwieriger Bruch in eine Summe von einfacheren
Brüchen zerlegt. Oft wird dies verwendet um anschliessend den Bruch einfacher integrieren zu können.

\begin{enumerate}
	\item Den Grad des Zählers und des Nenners vergleichen von $R$
	\begin{enumerate}
		\item Ist der Grad des Zählers $\geq$ der Grad des Nenners, so macht man eine 
		Polynomdivision. Man erhält daraus das Polynom $P$ und möglicherweise einen 
		Rest $R^*$, sodass gilt: $R = P + R^*$.
		\begin{enumerate}
			\item Ist $R^* \equiv 0$, so ist dieses Verfahren abgeschlossen.
			\item Sonst arbeitet man nun mit $R^*$ als Bruch weiter.
		\end{enumerate}
		\item Ist der Grad des Zählers $<$ der Grad des Nenners, so arbeitet man mit $R$ als Bruch weiter.
	\end{enumerate}
	\item Man berechnet die Nullstellen vom Nenner des Bruches (Mitternachtsformel/Raten). 
	Eine Nullstelle $x_0$ ist $r$-fach, wenn $f$ selbst und die ersten $r-1$ Ableitungen von $f$ an der Stelle $x_0$ den Wert $0$ annehmen und $f^{(r)}(x_0) \neq 0$.
	\item Nun setzt man den Bruch aus Schritt 2. gleich der Summe der
	Partialbrüche. Wie die Partialbrüche aussehen ist abhängig von den Nullstellen.  
	\begin{enumerate}
		\item Für jede einfache reelle Nullstelle $x_i$ ist der Summand
		$\frac{a_{i1}}{x-x_i}$ zu nehmen
		\item Für jede $r_i$-fache Nullstelle $x_i$ erhält man $r_i$ Summanden:
		$\frac{a_{i1}}{x-x_i} + \frac{a_{i2}}{(x-x_i)^2} + \ldots +
		\frac{a_{ir_i}}{(x-x_i)^{r_i}}$
	\end{enumerate}
	\item Nun berechnet man die unbekannten $a_{ij}$ indem man die Partialbrüche
	gleichnamig macht und dann die Koeffizienten des ursprünglichen Zählers mit
	denen des gleichnamigen Bruchs vergleicht.
\end{enumerate}

\subsection*{Beispiel}
$R(x) = \frac{x^2}{x^2-2x+1}$.

Der Zählergrad ist gleich dem Nennergrad,
weswegen wir eine Polynomdivision durchführen: $\Rightarrow R(x) = 1 +
\frac{2x-1}{(x-1)^2}$.

Aus $(x-1)^2$ folgt, das wir nur eine Nullstelle haben $x_0 = 1$. Es handelt
sich dabei um eine doppelte Nullstelle. Somit gilt:
\begin{align*}
\frac{2x-1}{(x-1)^2} &= \frac{a_1}{x-1} + \frac{a_2}{(x-1)^2}\\
2x-1 &= a_1(x-1) + a_2\\
2x-1 &= \ucomment{= 2x}{a_1 x} \;\; \ucomment{= -1 }{- a_1 + a_2}
\end{align*}
Daraus folgt, dass $a_1 = 2$ und $a_2 = 1$ (lin. Gleichungssystem).

Somit gilt: $R(x) = \frac{x^2}{x^2-2x+1} = 1 + \frac{2}{x-1} +
\frac{1}{(x-1)^2}$
