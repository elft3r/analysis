\section{Differentialrechnung in $\R^n$}
Hier geht es um Funktionen $f: \R^n \to \R^m$, wobei $m=1$ gelten kann
($f: \R^n \to \R$). Solche Funktionen haben die allgemeine Form:
$f(x) = f(x_1, x_2, x_3, \ldots, x_n) = \begin{pmatrix}
f_1(x_1, x_2, x_3, \ldots, x_n)\\
f_2(x_1, x_2, x_3, \ldots, x_n)\\
\ldots\\
f_m(x_1, x_2, x_3, \ldots, x_n)
\end{pmatrix}$

Für nahezu alle Eigenschaften gilt: Die Vektorfunktion $f: \R^n \to \R^m$ hat
eine bestimmte Eigenschaft, wenn jede einzelne ihrer Komponenten
($f_1, f_2, \ldots, f_m$) die besagte Eigenschaft besitzen. Das Problem liegt
neu also nicht im Wertebereich, sondern vor allem in Definitionsbereich.

\subsection{Norm}
Eine Norm auf $\R^n$ ist die Funktion $\|\cdot\|: \R^n \to \R$ mit den folgenden
Eigenschaften:
\begin{itemize}
	\item $\forall x \in \R^n: \|x\| \geq 0$
	\item $\forall x \in \R^n: \|x\| = 0 \Leftrightarrow x = \vec{0}$
	\item $\forall x \in \R^n, \alpha \in \R: \|\alpha x\| = |\alpha| \|x\|$
	\item $\forall x,y \in \R^n: \|x + y\| \leq \|x\|+\|y\|$
\end{itemize}

\subsection{Partielle Ableitung}
$\frac{\partial f}{\partial x} = f_{x} = $ partielle Ableitung von $f$ nach $x$. Dabei leitet man die Funktion nach $x$ ab und betrachtet die anderen Variabeln als Konstant. \\
Merke: partielle differenzierbarkeit impliziert nicht stetigkeit!

\begin{definition}A heisst die Ableitung von f (einfach jede Komponente nach jeder Variabel ableiten) und wir schreiben:
\[	
	A = Df(x) \gtext{oder} A = df(x)
\]
 \\
Bsp. Sei $f: \R^2 \to \R^3$ eine Funktion:\\
$f(x, y) = 
\left(
	\begin{array}{c}
		g_1(x, y) \\
		g_2(x, y) \\
		g_3(x, y)
	\end{array}
\right)$  
Dann ist: $Df(x, y) = 
\left(
	\begin{tabular}{c|c}
		\vspace{0.1cm}$\frac{\partial g_1}{\partial x}$ & $\frac{\partial g_1}{\partial y}$ \\
		\vspace{0.1cm}$\frac{\partial g_2}{\partial x}$ & $\frac{\partial g_2}{\partial y}$ \\
		$\frac{\partial g_3}{\partial x}$ & $\frac{\partial g_3}{\partial y}$
	\end{tabular}
\right)$
\end{definition}

\begin{satz}[Satz von Schwarz]
Ist $f$ nach $x$ und $y$ zweimal partiell differenzierbar und sind die gemischten
partiellen Ableitungen $f_{xy}$ und $f_{yx}$ stetig, so gilt: $f_{xy} = f_{yx}$.
\end{satz}