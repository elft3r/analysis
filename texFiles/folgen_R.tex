\section{Folgen in $\R$}
Eine Folge $a_n$ ist eine Funktion von $\N \backslash \{0\} \to \R$. Man nennt $(a_1, a_2, .. )$ Folgenglieder. Bsp. $a_n = 1/n \Rightarrow$  $a_1 = 1, a_2 = 1/2, ..$  

\subsection{Definitionen}
\begin{description}
  \item[konvergent] $\lim_{x \to \infty} a_n$ existiert \index{konvergent} (aus Konvergenz folgt Beschränktheit)
  \item[divergent] $\lim_{x \to \infty} a_n$ existiert nicht \index{divergent}
  \item[Nullfolge] $\lim_{x \to \infty} a_n = 0$ gilt \index{Nullfolge}
  \item[beschränkt] $\exists \, C_1, C_2 \in \R$, sodass gilt: $C_1 \leq a_n \leq C_2$ \hspace{0.3cm} oder\\
  $\exists \, C$ sodass gilt: $|a_n| \leq C$
  \item[unbeschränkt] falls $(a_n)$ nicht beschränkt ist. Unbeschränkte Folgen
  sind stets \underline{divergent}
  \item[monoton wachsend] $a_n \leq a_{n+1} \quad \forall n \in \N$ \index{monoton}
  \item[streng monoton wachsend] $a_n < a_{n+1} \quad \forall n \in \N$
  \item[monoton fallend] $a_n \geq a_{n+1} \quad \forall n \in \N$
  \item[streng monoton fallend] $a_n > a_{n+1} \quad \forall n \in \N$
  \item[alternierend] {\small die Vorzeichen der Folgenglieder wechseln sich ab}
  \item[bestimmt divergent / uneigentlich konvergent] es gilt $\lim_{n \to
  \infty} a_n = \pm \infty$
\end{description}

\begin{definition}[Konvergenz / Grenzwert] \index{Konvergenz / Grenzwert}
Die Folge $a_n$ konvergiert gegen a falls gilt:\\
\vspace{-0.3cm}\begin{align*}
\forall \epsilon > 0 \; \exists \, n_0 \in \N \; \forall n \geq n_0: |a_n - a| < \epsilon
\end{align*}
Wir nennen $a$ den Grenzwert/Limes und schreiben dann:
\begin{align*}
	\lim_{n \to \infty} a_n = a
\end{align*}
\end{definition}

\begin{definition}[Teilfolge] \index{Teilfolge}
Werden von einer Folge beliebig viele Glieder weggelassen, aber nur so viele,
dass noch unendlich viele übrigbleiben, so erhält man eine Teilfolge.
Konvergiert eine Folge $a_n$ gegen a, so konvergiert auch jede Teilfolge gegen $a$!
\end{definition}

\begin{definition}[Häufungspunkt] \index{Häufungspunkt} (Bolzano-Weierstrass)
	Ein Häufungspunkt ist ein Grenzwert einer Teilfolge (Bsp. H-Punkte von $(-1)^n$ = $\{1, -1\}$). \\
	Anderst ausgedrückt: 
	$a$ ist Häufungspunkt der Folge $(a_n)$, wenn in jeder Umgebung von $a$
	unendlich viele Folgeglieder liegen.
\end{definition}

\begin{definition}[Limes superior / Limes inferior] \index{$\limsup$}\index{$\liminf$}
	Ist $a_n$ eine beschränkte Folge so heisst der grösste Häufungspunkt Limes
	superior ($\limsup_{n \to \infty} a_n$ / $\overline{\lim}_{n \to \infty}
	a_n$) und der kleinste Häufungspunkt Limes inferior ($\liminf_{n \to
	\infty} a_n$ / $\underline{\lim}_{n \to \infty} a_n$)
\end{definition}

\subsection{Cauchy-Folgen}\index{Cauchy-Folge}
\begin{definition}
Sei $(a_n)_{n \in \N}$ eine Folge in $\R$. $(a_n)_{n \in \N}$ heisst \textbf{Cauchy-Folge}, falls gilt
\begin{align*}
\forall \epsilon > 0 \; \exists \, n_0 \in \N \; \forall n, l \geq n_0: |a_n - a_l| < \epsilon
\end{align*}
\end{definition}

Die Definition sagt grundsätzlich aus, dass ab einem $n_0$ (also einem Anfang $n_0$, der nur abhängig von $\epsilon$ ist)
die Folgeglieder nur noch $\epsilon$ Abstand zu einander haben. Also der Abstand beliebig klein wird zwischen Folgegliedern.

\begin{satz}[Cauchy-Kriterium]
Für $(a_n)_{n \in \N} \subset \R$ gilt:
\begin{align*}
	(a_n) \; \text{ist konvergent} \; \Leftrightarrow \; (a_n) \; \text{ist eine Cauchy-Folge}
\end{align*}
\end{satz}


\subsection{Rechnen mit Eigenschaften}
Addition:
{\small
\begin{itemize}
  \item $(a_n), (b_n)$ konvergiert $\Rightarrow (a_n + b_n)$ konvergiert
  \item $(a_n)$ konvergiert, $(b_n)$ divergent $\Rightarrow (a_n + b_n)$
  divergent
  \item $(a_n)$ beschränkt, $(b_n)$ beschränkt $\Rightarrow (a_n + b_n)$
  beschränkt
  \item $(a_n)$ beschränkt, $(b_n)$ unbeschränkt $\Rightarrow (a_n + b_n)$
  unbeschränkt
  \item $(a_n)$ beschränkt, $(b_n) \to \pm \infty \Rightarrow (a_n + b_n) \to
  \pm \infty$
  \item $(a_n) \to \infty$, $(b_n) \to \infty \Rightarrow (a_n + b_n) \to \infty$
  \item $(a_n) \to -\infty$, $(b_n) \to -\infty \Rightarrow (a_n + b_n) \to
  -\infty$
\end{itemize}
}

Produkt:
{\small
\begin{itemize}
  \item $(a_n)$ Nullfolge, $(b_n)$ beschränkt $\Rightarrow (a_n b_n)$ Nullfolge
  \item $(a_n)$ konvergent, $(b_n)$ beschränkt $\Rightarrow (a_n b_n)$
  beschränkt
  \item $(a_n)$ konvergent, $(b_n)$ konvergent $\Rightarrow (a_n b_n)$
  konvergent
  \item $(a_n)$ konvergent gegen $a \neq 0$, $(b_n)$ divergent $\Rightarrow
  (a_n b_n)$ divergent
\end{itemize}
}

\subsection{Rechnen mit Grenzwerten}
$\lim_{n \to \infty} a_n = a$, $\lim_{n \to \infty} b_n = b$\\
\emph{\underline{Achtung!} Untenstehendes gilt \underline{nur} wenn die Grenzwerte von $a_n$ und $b_n$ existieren. (Nicht $0$ oder $\infty$ sind.)}
\begin{itemize}
  \item $\lim_{n \to \infty} (a_n \pm b_n) = a \pm b = \lim_{n \to \infty} a_n + \lim_{n \to \infty} b_n$
  \item $\lim_{n \to \infty} (c \cdot a_n) = c \cdot a$
  \item $\lim_{n \to \infty} (a_n \cdot b_n) = a \cdot b = \lim_{n \to \infty} a_n \cdot \lim_{n \to \infty} b_n$
  \item $\lim_{n \to \infty} (a_n)^c = (\lim_{n \to \infty} a_n)^c, \quad$ nur wenn $c \neq n$
  \item $\lim_{n \to \infty} \frac{a_n}{b_n} = \frac{a}{b}, \quad$ nur wenn $(b_n)$ keine Nullfolge
\end{itemize}

\subsection{Hilfsmittel}
\textbf{Bernoullische Ungleichung}: Für $x \geq -1$ und $n \in \N$
\[
	(1+x)^n \geq 1 + nx
\]


\textbf{Vergleich von Folgen}: weiter rechts stehende Werte gehen schneller nach
$\infty$
\[
	1, \quad \ln n, \quad n^\alpha \; (\alpha > 0), \quad q^n \; (q > 1), \quad n!,
	\quad n^n
\]

\textbf{Stirlingformel}:
\[
	n! \approx \sqrt{2 \pi n} \left (\frac{n}{e} \right )^n
	\Rightarrow \left ( \frac{n}{e} \right )^n \sqrt{2 \pi n} \leq n! \leq \left (
	\frac{n}{e} \right )^n \sqrt{2 \pi n} \cdot e^\frac{12}{n}
\]

\subsection{Konvergenzkriterien}\index{Konvergenzkriterien}
\begin{align*}
	a_n \to a \Leftrightarrow a_n - a \to 0 \Leftrightarrow |a_n - a| \to 0
\end{align*}
	
\begin{itemize}[leftmargin=*]	
	\item Ist $\lim_{n \to \infty} a_n = a$, so ist der Limes $a$ einziger
	Häufungspunkt der Folge $(a_n)$ und jede Teilfolge konvergiert auch gegen $a$.
	
	\textbf{Beispiel:} Wegen $\left( 1 + \frac{1}{n} \right)^n \to e$, so gilt auch
	$\left( 1 + \frac{1}{2n} \right)^{2n} \to e$
	
	\item \textbf{Divergenzkriterium:} Hat die Folge zwei verschiedene Häufungspunkte, so ist die Folge sicher
	divergent.
	
	\item \textbf{Monotone konvergenz:} Ist die Folge monoton steigend (fallend) und nach oben (unten) beschränkt, dann konvergiert $\lim_{n \to \infty} a_n$ zu $\sup \; a_n$ $(\inf \; a_n)$
	
	\item Wenn $\sum_{n=0}^\infty a_n$ konvergiert, so ist $\lim_{n \to \infty} a_n = 0$\\
	Damit kann man die Grenzwertregeln für Reihen verwenden.
	
	\item Gibt es eine Funktion $f$ mit $f(n) = a_n$ und $\lim_{x \to \infty} f(x)
	= a$, so gilt auch $\lim_{n \to \infty} a_n = a$.\\
	Damit kann man zum Beispiel die Regel von \underline{l'Hospital} und die
	restlichen Methoden anwenden. Siehe Grenzwerte von Funktionen.
	\underline{Achtung:} Es kann sein, dass $f$ keinen Grenzwert besitzt, aber
	$(a_n)$ schon.
	
	\item \textbf{Einschliessungskriterium}: Sind $(a_n), (b_n), (c_n)$ Folgen mit
	$a_n \leq b_n \leq c_n$ und haben $(a_n), (c_n)$ den gleichen Grenzwert $a$, so
	konvergiert auch $(b_n)$ nach $a$.

	\item \textbf{Cauchy-Kriterium:} \\
	Die Folge $a_n$ ist konvergent $\Leftrightarrow$ $a_n$ ist eine Cauchy-Folge 
\end{itemize}

\subsection{Konvergenztipps \& Beispiele}
\subsubsection{Faktoren klammern und kürzen / Brüche}
Bei Brüchen und auch sonst empfiehlt es sich oft den am stärksten wachsenden Teil (das am
schnellsten wachsende $n$) zu kürzen. In diesem Fall ist es das $n^4$ in der
Wurzel, also $n^2$.
\vspace{0.1cm}
\begin{align*}
\lim_{n \to \infty} \frac{n^2 + \ln n}{\sqrt{n^4 - n^3}} 
&= \lim_{n \to \infty} \frac{n^2 + \ln n}{\sqrt{n^4 - n^3}} 
\cdot \frac{\frac{1}{n^2}}{\frac{1}{n^2}} = \lim_{n \to \infty} \frac{n^2 + \ln
n}{n^2 \sqrt{1 - \frac{1}{n}}} \cdot \frac{\frac{1}{n^2}}{\frac{1}{n^2}} \\
&= \lim_{n \to \infty} \frac{1 + \frac{\ln n}{n^2}}{\sqrt{1 - \frac{1}{n}}}
= \frac{1 + 0}{\sqrt{1 - 0}} = 1
\end{align*}

\subsubsection{l'Hospital für Folgen (Folge als Funktion)}
$\lim_{n \to \infty} \frac{\ln n}{n^2}$

Die Funktion $f(x) = \frac{\ln x}{x^2}$ entspricht unseren Folgegliedern ($f(n)
= a_n = \frac{\ln n}{n^2}$). Für $n \to \infty$ hat der Nenner und der Zähler
den Grenzwert $\infty$, also wenden wir die Regel von l'Hospital an.

\begin{align*}
\ldots &= \lim_{x \to \infty} \frac{(\ln x)'}{(x^2)'} = \lim_{x \to \infty}
\frac{\frac{1}{x}}{2x} = \lim_{x \to \infty} \frac{1}{2x^2} = 0
\end{align*}

Somit geht auch die Folge gegen 0.

\subsubsection{Wurzeln}
$\lim_{n \to \infty} (\sqrt{n^2 + an + 1} - \sqrt{n^2 + 1})$

Die einzelnen Terme streben jeweils gegen unendlich und \\
$(\infty - \infty)$ kann nicht direkt berechnet werden. Deshalb macht man hier eine Brucherweiterung \underline{mit geändertem Vorzeichen} in der Mitte!
\begin{align*}
&= \lim_{n \to \infty} (\sqrt{n^2 + an + 1} - \sqrt{n^2 + 1}) \cdot
\left(\frac{\sqrt{n^2 + an + 1} + \sqrt{n^2 + 1}}{\sqrt{n^2 + an + 1} +
\sqrt{n^2 + 1}} \right) \\
&= \lim_{n \to \infty} \frac{(n^2 + an + 1) - (n^2 + 1)}{\sqrt{n^2 + an + 1} +
\sqrt{n^2 + 1}} \\
&= \lim_{n \to \infty} \frac{an}{\sqrt{n^2 + an + 1} + \sqrt{n^2 + 1}} 
\end{align*}

nun verwenden wir den Tipp für Brüche und kürzen das $n$ heraus

\begin{align*}
\ldots &= \lim_{n \to \infty} \frac{a}{\sqrt{1 + \frac{a}{n} + \frac{1}{n^2}} +
\sqrt{1 + \frac{1}{n^2}}} = \frac{a}{1 + 1} = \frac{a}{2}
\end{align*}

\subsubsection{Laufvariable im Exponent}
$\lim_{x \to 0} (3 - |x|)^{\frac{\sin(x)}{x}}$\newline
$\Rightarrow (3 - |x|)^{\frac{\sin(x)}{x}} = e^{ln((3 - |x|)^{\frac{\sin(x)}{x}})}$\newline
$^{\text{Verwende } (y = e^{\ln y})}$\newline
$\Rightarrow e^{\frac{\sin(x)}{x} \cdot \ln(3 - |x|)}$\newline
$\Rightarrow \lim_{x \to 0}\frac{\sin(x)}{x}\cdot \ln(3-|x|) = \ln(3)$\newline
$^{\text{(Berechne Limes vom Exponenten separat)}}$\newline
$\Rightarrow \lim_{x \to 0}e^{\frac{\sin(x)}{x} \cdot \ln(3 - |x|)} = e^{\ln(3)} = 3$\newline
$^{\text{(Setze Limes im Exponent ein)}}$

\subsubsection{Term erweitern}
Oft steht ein Term da, der annähernd so aussieht wie etwas das wir bereits kennen. Durch Term/Bruch - Erweiterung lässt er sich oft auf eine bekannte Form bringen, welches separat gelöst werden kann.
\[
	\lim_{x \to 0} \frac{\sin |x|}{\sqrt{|x|}} = 
	\lim_{x \to 0+} \frac{\sqrt{x} \cdot \sin x}{\sqrt{x} \cdot \sqrt{x}} =
	\lim_{x \to 0+} \sqrt{x} \cdot \frac{\sin x}{x} = 0 \cdot 1 = 0
\]

\subsubsection{Einschliesskriterium}
Vereinfache $a_n$ zu $b_n$ und $c_n$ so dass gilt: $b_n \leq a_n \leq c_n$. \\
Die Grenzwerte von $b_n$ und $c_n$ sollten einfach auszurechnen sein! 
Wenn $b_n$ und $c_n$ gegen $a$ strebt, dann macht dies auch $a_n$.

\subsubsection{Gruppieren}
%Aus Analysis II (2012) Serie 1 Aufgabe 2b
Hat man zum Beispiel die Folge $s_n = 1 + \frac{1}{3} + \frac{1}{5} + ... + \frac{1}{2n-1}$ so kann man die 
einzelnen Therme gruppieren und abschätzen in diesem Fall z.B. mit $\frac{1}{4}$ (zweiter Therm, dritter und vierter Therm,
fünfter bis achter Therm, nächste 8 Therme, etc.). Wir erhalten dann $s_{2^k} \geq 1 + \frac{k}{4}$ und sehen somit das die 
Reihe nicht beschränkt ist und somit auch nicht konvergiert.