\section{Funktionsfolgen}
Wenn $f_n : \Omega \to \R$ eine Funktion ist (für jedes $n \in \N$) dann nennt man $f_n$ eine Funktionenfolge. Dh: Eine Folge von Funktionen.

\begin{definition}[punktweise konvergent] \index{punktweise konvergent}
Die Funktionenfolge $f_n$ \underline{konvergiert punktweise} gegen $f$, falls gilt:
\[
	\lim_{n \to \infty} f_n(x) = f(x) \hspace{1cm} \text{für} \; \forall x \in \Omega
\]
\[
\text{Formal: } \forall \epsilon > 0 \; \forall x \in \Omega \; \exists n_0 \in \N: n \geq n_0
\Rightarrow |f(x) - f_n(x)| < \epsilon
\]
\end{definition}


\begin{definition}[gleichmässig konvergent] \index{gleichmässig konvergent}
Die Funktionenfolge $f_n$ \underline{konvergiert gleichmässig} gegen $f$, falls gilt:
\[
	\lim_{n \to \infty} \sup_{x \in \Omega} |f_n(x) - f(x)| = 0 
\]
\[
\text{Formal: } \forall \epsilon > 0 \; \exists n_0 \in \N \; \forall x \in \Omega: n \geq n_0
\Rightarrow |f(x) - f_n(x)| < \epsilon
\]
Das bedeutet, dass die obere Definition für alle $x$ \underline{dasselbe} $n_0$ verwendet und nicht jeweils
verschiedene!\\
\textbf{Merke:}\\
gleichmässige Konvergenz $\Rightarrow$ punktweise Konvergenz.\\
$f(x)$ nicht stetig $\Rightarrow$ $f_n$ konvergiert nicht gleichmässig gegen $f(x)$
\end{definition}

\textbf{Beispiel:} Zeige dass $f_n(x) = 1 + x^n(1 - x)^n$ auf dem Intervall $[0, 1]$ gleichmässig konvergiert.
\begin{enumerate}
\item Wir zeigen dass $f_n$ Punktweise stetig ist.
\[
	\lim_{n \to \infty} f_n(x) = \lim_{n \to \infty} 1 + x^n(1 - x)^n = 1 = f(x) \text{  weil $x \in [0, 1]$}
\]

\item Wir zeigen dass $f_n$ auch gleichmässig stetig ist.
\begin{flalign*}
	\hspace{-0.4cm}&*\text{Abschätzen: } x(1-x) \leq 0.5(1 - 0.5) = \frac{1}{4} \text{ für } \forall x \in [0, 1]&\\
	\hspace{-0.4cm}&|f_n(x) - f(x)| = | 1 + x^n(1 - x)^n - 1| = |\left(x(1 - x)\right)^n| \overset{*}{\leq} \left(\frac{1}{4}\right)^n&\\
	\hspace{-0.4cm}&\text{Somit gilt:} \lim_{n \to \infty} \sup_{x \in [0, 1]} |f_n(x) - f(x)| \leq \lim_{n \to \infty} \left(\frac{1}{4}\right)^n = 0&
\end{flalign*}
\end{enumerate}